\documentclass{beamer}
\mode<presentation>

\title{There Is No Largest Prime Number}
\author[Euclid]{Euclid of Alexandria \\ \texttt{euclid@alexandria.edu}}
\date[ISPN ’80]{27th International Symposium of Prime Numbers}

\begin{document}:
\begin{frame}
\titlepage
\end{frame}

\begin{frame}
\frametitle{Outline}
\tableofcontents[pausesections]
\end{frame}

\section{Motivation}
\subsection{The Basic Problem That We Studied}

\begin{frame}
\frametitle{What Are Prime Numbers?}
\begin{definition}
A \alert{prime number} is a number that has exactly two divisors.
\end{definition}
\begin{example}
\begin{itemize}
\item 2 is prime (two divisors: 1 and 2).
\item 3 is prime (two divisors: 1 and 3).
\item 4 is not prime (\alert{three} divisors: 1, 2, and 4).
\end{itemize}
\end{example}
\end{frame}

\begin{frame}[fragile]
\frametitle{An Algorithm For Finding Primes Numbers.}
\begin{verbatim}
int main (void)
{
	std::vector<bool> is_prime (100, true);
	for (int i = 2; i < 100; i++)
		if (is_prime[i])
		{
			std::cout << i << " ";
			for (int j = i; j < 100; is_prime [j] = false, j+=i);
		}
	return 0;
}
\end{verbatim}
\begin{uncoverenv}<2>
Note the use of \verb|std::|.
\end{uncoverenv}
\end{frame}
\end{document}