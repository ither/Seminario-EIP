% !TeX spellcheck = es_AR
\documentclass{beamer}
\usepackage{lmodern}
\usepackage{listings}
\usepackage[T1]{fontenc}
\usepackage[utf8]{inputenc}
\usepackage[spanish]{babel}

\mode<presentation>
{
  \usetheme{Warsaw}
  \setbeamercovered{transparent}
}

\title{Patrones de integración}
\subtitle{Apache Camel y Spring Integration}
\author[Despegar.com]{Pablo Rochás\\ \texttt{pablo.e.rochas@gmail.com}}
\date[Despegar]{Junio 2014}

\begin{document}

\begin{frame}
\titlepage
\end{frame}

\begin{frame}{Esquema}
  \tableofcontents
\end{frame}

\section{Introducción}
\subsection{El problema}
\begin{frame}{Integración de sistemas}
¿A qué llamamos integración de sistemas?
\end{frame}

\begin{frame}{El problema}
Problemas de integración al escribir una aplicación
\begin{itemize}
\item Distintas aproximaciones por programador
\item Cambios constantes y deuda técnica
\item La experiencia de Ant
\end{itemize}
\end{frame}

% tapa del libro de EIP
% Se edito en agosto 2004
% los frameworks en _tal_ año
% Diferencias entre las implementaciones:
% - Spring mucho mas cercano al libro
% - Camel mas cómodo y flexible para el programador, se ajusta a su manera de programar
% El ejemplo se desarrolla con spring
% no es la panacea
% preambulo ejemplo: endpoint, channel, message
% ejemplo mail sender
% casos especiales: excepciones, transacciones
% endpoints y/o canales interesantes para destacar (jdbc inbound ch adapter, jpa outbound gateway, etc)
% subscriptions api v3
% wiretap, gateway con java
% conclusiones: no es necesario usar una implementacion, se trata de aprender el lenguaje comun


%\section{Introduction}
%\subsection[Short First Subsection Name]{First Subsection Name}
%\begin{frame}{Make Titles Informative. Use Uppercase Letters.}{Subtitles are optional.}
%\end{frame}
\end{document}